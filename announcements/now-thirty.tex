% Created 2024-07-27 Sat 13:25
% Intended LaTeX compiler: pdflatex
\documentclass[11pt]{article}
\usepackage[utf8]{inputenc}
\usepackage[T1]{fontenc}
\usepackage{graphicx}
\usepackage{longtable}
\usepackage{wrapfig}
\usepackage{rotating}
\usepackage[normalem]{ulem}
\usepackage{amsmath}
\usepackage{amssymb}
\usepackage{capt-of}
\usepackage{hyperref}
\usepackage[scaled]{helvet} \renewcommand\familydefault{\sfdefault}
\input{.aster.tex}%
\author{T.V Raman}
\date{Thursday, August 1, 2024}
\title{Emacspeak:  A Speech Odyssey}
\hypersetup{
 pdfauthor={T.V Raman},
 pdftitle={Emacspeak:  A Speech Odyssey},
 pdfkeywords={Emacspeak, Complete Audio Desktop},
 pdfsubject={Emacspeak  --- A Speech Odyssey --- Emacspeak At Thirty},
 pdfcreator={Emacs 31.0.50 (Org mode 9.7.6)}, 
 pdflang={English}}
\begin{document}

\maketitle
\tableofcontents

\section{Dedication: To My Guiding Eyes}
\label{sec:org58050b6}
\section{Key Insights}
\label{sec:org50c9613}
\begin{enumerate}
\item \href{https://www.drdobbs.com:443/user-interface-a-means-to-an-end/184410453}{User interface is a means to an end}.
\item Open Source is essential  for discovering new interaction paradigms.
\item This is not mere idealism.  Openness is a key enabler for
creating   user journeys that were not  envisioned by a 
system's designers.
\item \href{https://www.gnu.org/s/emacs/}{Emacs} and  \href{https://en.wikipedia.org/wiki/TeX}{\TeX{}}    are good exemplars. They  permit maximal freedom
 when seen from the viewpoint of user extensibility and
creativity. \TeX{} enabled \href{https://emacspeak.blogspot.com/2022/12/aster-spoken-math-on-emacspeak-audio\_21.html}{Audio System For Technical Readings (AsTeR)}; Emacs enabled \href{https://emacspeak.sourceforge.net}{Emacspeak}.
\item Rapid, reliable task completion is the most important metric and
trumps secondary items such as eye-candy --- the latter only
leads to bloat as evinced by the HTML Web.
\item Having a well-identified  problem when designing a system
is paramount.
\item \emph{Usability} is important, but to   matter, the
system needs to be \emph{useful} first.
\item \emph{Ease of use} by   itself is often \emph{marketing} hype.
\item Useful systems are fun to learn and give back more than what you put
in with respect to time and effort.
\item A steep learning curve in and of itself is not to be feared --- it
can be fun to learn and  gets you farther faster.
\item True empowerment: Ensure that the user grows continuously.
\end{enumerate}
\section{Emacspeak --- The Complete Audio Desktop}
\label{sec:org8a092cb}

\begin{enumerate}
\item Emacspeak, started in September 1994, was released as Open
Source in \href{https://tvraman.github.io/emacspeak//web/releases/release-3.0.html}{April 1995}.
\item The goal was to create a system for daily use that  doubled
as a research work-bench for developing an auditory interface.
\item Speech and auditory output would be  treated as 
first-class citizens.
\item The time felt right with respect to building a  system 
that enabled  eyes-free access to the emerging Web.
\item \href{https://emacspeak.sourceforge.net/turning-twenty.html}{Emacspeak At Twenty}  was published in September 2014 and  traced the
evolution of the project.
\item Now, this article gives a birds-eye overview of the last 10 years
by loosely following the logical structure of the  \emph{Turning Twenty} paper.
\item In the process, we identify the dreams that have come to pass as
well as the expectations that have failed to materialize --- \textbf{both}
attributable  to developments in the larger Internet eco-system.
\item But never fear, though  some of these
may be   superficially
disappointing, they likely herald the nature of bigger and better
things to come!
\item As a proof-point, in 1994, I could not have imagined the impact
that the world of Internet-centered  computing and the accompanying
information revolution would have  on the state of information
access.
\item Conversely, I boldly (and incorrectly) predicted that the
arrival of mobile devices and mainstream speech interfaces would
herald the move to a Web of information where there would be a
clean separation between application back-ends and various
client-specific front-ends. See \href{https://emacspeak.sourceforge.net/raman/publications/specialized-browsers/}{Specialized Browsers} and \href{http://www.cs.washington.edu/htbin-post/mvis/mvis?ID=636}{The
Web, The Way You Want.  Distinguished Lecture Series, UW Oct 2007}.
\item The above still makes sense from the view of  \emph{scalable} software architecture. However the rapid growth of the Web economy has also
resulted in an even faster race to the bottom where applications
continue to be built and re-built every few years for \emph{the next
best thing} --- welcome to the \emph{write once, debug everywhere}
world all over again!
\item Case in point; today we have smart phones, smart watches  and smart speakers,
but each of these  require targeted front-ends  
if one wishes to  bring the riches of the Internet to them.
\item So the larger the Web gets, the fewer devices it becomes
available  on --- a classic downward spiral.
\end{enumerate}

Share And Enjoy --- \href{https://tvraman.github.io/emacspeak/web/01-gemini.ogg}{The Best Is Yet To Come}!
\subsection{How To Read This Document}
\label{sec:org1bfdd34}

\begin{enumerate}
\item I recommend reading the \emph{Turning Twenty} paper to get a full overview.
\item Then, read this paper a section at a time, while referring back to
the parallel section in the \emph{Turning Twenty} paper to understand
how things have evolved.
\item Make sure to skim or deep-dive into the references in both papers.
\end{enumerate}
\section{Using UNIX With Speech Output —  2024}
\label{sec:orge07ecee}

\begin{enumerate}
\item In 2024 \emph{UNIX} equates mostly to various Linux distributions, and from
the Emacspeak perspective, they are all made \emph{mostly} equal.
\item Variations do exist and  running bleeding-edge distributions can come
with issues, \emph{e.g.}, unstable versions of the underlying audio infrastructure.
\item Yes, 30 years and counting, Linux Audio is still a work in
progress though I hope Pipewire will be the last of these tidal shifts.
\item Linux is moving to Wayland  and expect that transition to
be choppy.
\item Native applications are mostly gone bar
the shouting. In this context, where most users access things
through a \emph{mainstream} Web browser, Emacspeak users access
\emph{everything} through Emacs.
\item The above when  done right is hugely empowering; 
 when done badly, it's extremely limiting  --- see later
sections of this paper on  the continuing evolution of the Web.
\end{enumerate}
\section{Key Enabler — Emacs And Lisp Advice}
\label{sec:orgaff3f05}

\begin{enumerate}
\item Advice in Emacs as implemented in \texttt{advice} is rock-solid.
\item There is a newer \texttt{nadvice} that is part of Emacs that Emacspeak
does not use.

\item There are no plans to migrate to \texttt{nadvice} since that is a lot of
busy work in my view and any such migration would be difficult
to test for correctness.
\item The classic \emph{advice} package may be removed from Emacs at some
point in the future, but never fear; it'll be bundled with
Emacspeak if that becomes necessary. This is a feature of Free Software and is a great
example of what that \emph{Freedom} entails.
\end{enumerate}
\section{Key Component —  Text To Speech (TTS)}
\label{sec:orga7888b2}

\begin{enumerate}
\item Speech output --- especially unencumbered text-to-speech --- is just
as much a challenge as it was 30 years ago.
\item In the bigger picture, early instances of using TTS for voice
assistants has driven the industry toward \emph{natural sounding} voices.
\item The above sounds attractive on the surface, but a price we have
paid is the  loss of fine-grained control over voice parameters,
emotion, stress and other supra-linguistic features.
\item I  believe  these to be essential for delivering
good auditory interfaces and   remain optimistic that
these will indeed arrive in a future iteration of speech
interaction.
\item Things appear to be coming full circle, Emacspeak started with
the hardware Dectalk; now, the \href{https://github.com/dectalk/dectalk.git}{Software Dectalk} is increasingly
becoming the primary choice on Linux --- see this  \href{https://raw.githubusercontent.com/tvraman/emacspeak/master/servers/software-dtk/Readme.org}{Readme for setup instructions}.
\item Viavoice Outloud from Voxin is still supported.  However,
you can no longer buy new licenses. If you have already purchased
a license, it'll
continue to work.
\item The  Vocalizer voices that Voxin now sells \emph{do not} work with Emacspeak.
\item The  other choice on Linux is ESpeak which will hopefully
continue to be free --- albeit of much lower quality.
\item The future as ever is unpredictable and new voices may well show
up --- especially those powered by on-device Large Language
Models (LLMs).

\item On non-free platforms, there is usable TTS on the Mac, now
supported by the new SwiftMac server for Emacspeak.
\end{enumerate}
\section{Emacspeak And Software Development}
\label{sec:org43896d0}

\begin{enumerate}
\item \emph{Magit}  as a Git porcelain is perhaps the biggest leap forward
with respect to software development.
\item New completion frameworks such as \emph{company} and \emph{consult} come a
close second in enhancing productivity.
\item Completion strategies such as  \emph{fuzzy} and
\emph{flex} provide  enhanced completion.
\item \href{https://emacspeak.blogspot.com/2018/06/effective-suggest-and-complete-in-eyes.html}{Effective Suggest And Complete In An Eyes-free Environment}
explains the higher-level concept  involved in defining such strategies.

\item The ability to introspect code via  \texttt{eglot} 
 turn Emacs into a powerful and meaningful IDE ---  I say
meaningful because this brings the best features of an integrated
development environment while leaving behind the eye-candy that
tends to bloat commercial IDEs.
\item Packages like \emph{transient}  enable discoverable, rapid keyboard access to
complex nested-menu driven interfaces.
\item \href{https://emacspeak.blogspot.com/2023/09/emacs-ergonomics-dont-punish-your.html}{Ergonomic keybindings} under \texttt{X} using \href{https://github.com/alols/xcape}{xcape} to minimize
chording has been  a significant win in the last two years.
\item Jupyter is the   generalization of IPython notebooks to \emph{Julia}, \emph{Python}
and \emph{R}. The news here isn't all good; IPython notebooks are
well-designed with respect to not getting locked into any given
implementation. However in practice,  front-ends
depend on Javascript in the  browser.
\item Consequently,  Emacs  packages  for IPython
Notebooks \emph{e.g.}, package \texttt{ein},  are no longer maintained.
\item Developing in higher-level languages continues to be very well
supported in Emacspeak.
\item The re-emergence of Common Lisp in the last 20 years, thanks to
\href{https://asdf.common-lisp.dev/asdf.html}{asdf} and \href{https://www.quicklisp.org/}{quicklisp} as a network-aware package manager and build
tool has once again made Lisp development using Emacs \texttt{Slime} a
productive experience.
\item In 2022, I updated \href{https://emacspeak.blogspot.com/2022/12/aster-spoken-math-on-emacspeak-audio\_21.html}{Audio System For Technical
Readings (AsTeR)}--- my PhD project from 1993 --- to run under \texttt{SBCL}
with a freshly implemented Emacs front-end.
\item So now I can listen to Math content just as well as I could 30
years ago!
\end{enumerate}
\section{Emacspeak And Authoring Documents}
\label{sec:org5688109}

\begin{enumerate}
\item Package \texttt{org} is to authoring as \texttt{magit} is to
software development with respect  to productivity gains.
\item \texttt{Org} has existed since circa 2006 in my Emacs setup; but it
continues to give and give plentifully.
\item Where I once authored technical papers in \emph{\LaTeX{}} using \texttt{auctex},
used \texttt{nxml} for
HTML,  \emph{etc.}, I now mostly write everything in \texttt{org-mode} and export
to the relevant target format.
\item Integrating various search engines  in Emacs makes authoring content extremely productive.
\item Integrated access to spell-checking (\texttt{flyspell}) dictionaries, translation engines, and other
language tools combine for a powerful authoring work-bench.
\item Extending \texttt{org-mode} with custom link types enables \emph{smart note
taking} with hyperlinks to relevant portions of an audio stream
--- see article \href{https://emacspeak.blogspot.com/2022/10/learn-smarter-by-taking-rich-hypertext.html}{Learn Smarter By Taking Rich Hypertext Notes}.
\end{enumerate}
\section{Emacspeak  And The Web In 2024}
\label{sec:orgd4f1fb5}


\begin{enumerate}
\item Package \texttt{shr} and \texttt{eww} arrived around 2014. But in 2024, they
can be said to have \textbf{truly} landed.
\item 2014 also  marked the explicit take-over of the stewardship of the HTML Web by the
browser vendors from the W3C  --- I say
explicit ---  because the W3C had already thrown in the towel in the
preceding decade.
\item This  has led to a Web of content  created using the assembly
language of divs, spans and Javascript  under the flag of HTML5 ---
the result is a tangled web of spaghetti that everyone loves to hate.
\item In this context, see \href{https://idlewords.com/talks/website\_obesity.htm}{Tag Soup, Scripts And Obfuscation: How The
Web Was Broken} for  a good overview of  HTML's obesity problem.
\item For better or worse, the investment in XML and display-independent
content is now a complete write-off at least on the surface.

\item So what next --- wait for the spaghetti monster to show up for
lunch? Humor aside that monster may well be called AI ---  though
whether  today's Web gives that monster life, indigestion,
constipation,   dysentery or hallucinations  is a story to be
written in the coming years.

\item I say \emph{on the surface} above  because The welcome re-emergence of
\texttt{ATOM} and \texttt{RSS} feeds is perhaps a silent acknowledgement that
bloated Web pages are now unusable even for users who can see.
\item Package  \texttt{elfeed} has emerged as  a powerful feed-manager for Emacs.
\item Emacspeak implements  \texttt{RSS} and \texttt{ATOM} support using
\texttt{XSLT};  those features now shine brighter  with mainstream
news  sites reviving their support for content feeds.
\item Browsers like Mozilla now implement \emph{content filters} --- a
euphemism for scraping off  visual eye-candy and related cruft to
reveal the underlying content.  These are now 
available as  plugins, (see \href{https://github.com/eafer/rdrview}{RDRView} for an example).  Emacspeak 
leverages this to make the Web more readable.
\item Package \texttt{url-template} and \texttt{emacspeak-websearch} continue to give
in plenty, though they do require continuous updating.
\item Web APIs come and go, so 
that space is in  a state of constant change.
\item The state of web applications is perhaps the most concerning from an
Emacspeak perspective, and I do not  see that changing in the
short-term.  There are no incentives for Web providers to
free their applications from the tangled Web of spaghetti they have woven
around themselves.
\item But as with everything else in our industry,
it is precisely when something feels completely entrenched that users
rebel and innovations emerge  to move us to the next phase --- so
fingers crossed.
\end{enumerate}
\section{Audio Formatting —  Generalizing Aural CSS}
\label{sec:org221393b}

\begin{enumerate}
\item Audio formatting with Aural CSS support is stable, with new
enhancements supporting more TTS engines.
\item Support for parallel streams of TTS using separate outputs to
left/right channels is a big win and enables more efficient interaction.
\item Support for various Digital Signal Processing (DSP)  filters enables   rich auditory effects
like  binaural audio and spatial audio.
\item \href{https://emacspeak.blogspot.com/2015/12/soundscapes-on-emacspeak-audio-desktop.html}{Soundscapes} implemented via package \texttt{boodler} makes for  a
pleasant and relaxing auditory environment.
\item Enabling virtual sound devices via Pipewire for 5.1  and 7.1
spatial audio significantly enhances the auditory experience.
\end{enumerate}
\section{Conversational Gestures For The Audio Desktop}
\label{sec:orgc5128cf}

\begin{enumerate}
\item Parallel streams of audio, combined with more ergonomic
keybindings are  the primary  enhancement in this area.
\item Parallel streams of speech, \emph{e.g.}, a separate notification
stream on the left or right ear  help increase the band-width of communication.
\item Notifications can thus be delivered without having to stop the
primary speech output.
\end{enumerate}
\section{Accessing Media Streams}
\label{sec:org124c5df}


\begin{enumerate}
\item Emacspeak support for rich multimedia is now much  more robust.
\item Emacs package   \texttt{empv}  is a
powerful tool  for locating, organizing  and playing local and remote
media streams ranging from music, audio books, radio stations and
Podcasts.
\item This makes media streams from a large number of providers ranging
from the BBC to Youtube available via a consistent keyboard interface.
\item This experience is augmented by a collection of \emph{smart} content
locators on the Emacspeak desktop, see the relevant blog
article titled 
\href{https://emacspeak.blogspot.com/2024/03/updated-smart-media-selector-for-audio.html}{smart media selectors}.
\end{enumerate}
\section{Electronic Books—   Ubiquitous Access To Books}
\label{sec:org53657f1}

\begin{enumerate}
\item Emacspeak modules  for \emph{Epub} and 
\emph{Bookshare} continue to provide good books  integration.
\item There are  \emph{smart} book locators analogous to the locators for
media content.
\item Emacspeak speech-enables \texttt{Calibre}  for working with
local electronic libraries.
\end{enumerate}
\section{Leveraging Computational Tools —  From SQL And R To IPython Notebooks}
\label{sec:org3b660bb}

\begin{enumerate}
\item This area continues to provide a rich collection of  packages.
\item Newer highlights include \texttt{sage} interaction for symbolic computation.
\item Emacspeak speech-enables  packages  \texttt{gptel} and \texttt{ellama} for working
with local and network LLMs.
\end{enumerate}
\section{Social Web  — Mail, Messaging And Blogging}
\label{sec:orge69b250}

\begin{enumerate}
\item This is a space that is definitely regressing.
\item The previous decade was marked by open APIs to many social Web platforms.
\item Over time these first regressed with respect to privacy.
\item Then they turned into wall-gardens in their own right.
\item Finally, the Web APIs, other than the kind embedded in Javascript have
started disappearing.
\item Looking back, the only \emph{social} platform I now use is Blogger for
hosting my Emacspeak Blog, it has a somewhat usable API, albeit
guarded by a difficult to use \emph{OAuth} interface that requires 
signing   in via  a \emph{mainstream} browser.
\item IMap continues to survive as an open email protocol, though its
days may well be numbered.
\item The dye is already cast with respect to mere mortals being able
to setup and  host their email ---  witness the complexity in setting
up the Emacspeak mailing list in 2023 vs 1993!
\item This is an area that is  likely to get worse before it gets
better,  thanks to the spammers  --- more's  the pity, since Internet Email is perhaps the
single-most impactful technology with respect to leveling the
communications playing field.
\item The disappearance of APIs mentioned above also means that today
the only usable chat service on an open platform like Emacspeak
is the venerable  Internet Relay Chat (IRC).
\end{enumerate}
\section{The RESTful Web —  Web Wizards And URL Templates For Faster Access}
\label{sec:org34d79ba}

\begin{enumerate}
\item This area continues to thrive --- either because of --- or
despite --- the best and worst efforts of application providers on the
Web.
\item Twenty years on (this feature originally landed in 2000)
Emacspeak has a far richer collection of filters, preprocessors
and post-processors
 that enables ever-more powerful Web
wizards. See the relevant \href{https://tvraman.github.io/emacspeak/manual/URL-Templates.html}{chapter} in the Emacspeak manual for the
automatically updated list of \textbf{URL Templates}.
\end{enumerate}
\section{Mashing It Up —  Leveraging  AI And The Web}
\label{sec:org634b722}

\begin{enumerate}
\item Developing solutions by combining various API-based services on
the Web has all but disappeared, unless one is willing to commit
fully to the Javascript-powered Web hosted in a Web browser,
something I hope I never have to accept.
\item So for now, I'll keep
well away and count my blessings.
\item The next chapter of the \emph{mash-up story} may well be based around
 \emph{Generative AI} using LLMs. In effect, LLMs trained on   Web content 
define a \emph{platform} for generating content mash-ups.  The issue
at present is that they are just as  likely  to produce
\emph{meaningless mush} ---
something that may  get better as the field gets a
handle on cleaning up  Web content.
\item Notice that we are now back to the previously unsolved problem
of cleaning up the  HTML Web --- with LLMs, we'll just
have an order of magnitude more documents than the \emph{2\textsuperscript{W}} postulated
 by  \href{https://research.google/blog/beyond-web-20/?hl=in\&m=1}{Beyond Web 2.0, Communications
Of The ACM, 2009}.
\end{enumerate}
\section{The Final Word --- Donald E Knuth (DEK)}
\label{sec:org273aa86}

\begin{itemize}
\item The best theory is inspired by practice. The best practice is
inspired by theory.
\item The enjoyment of one's tools is an essential ingredient of
successful work.
\item Easy things are often amusing and relaxing, but their value soon
fades. Greater pleasure, deeper satisfaction, and higher wages are
associated with genuine accomplishments, with the successful
fulfillment of a challenging task.
\item \href{https://www.azquotes.com/author/8177-Donald\_Knuth}{Computer Programming Is An Art}.
\end{itemize}

The best example of the above is of course \href{https://en.wikipedia.org/wiki/TeX}{Knuth's \TeX{}} --- work that
    was motivated  by his own dissatisfaction with the tools available
    to him at the time for typesetting    his magnum opus --- \href{https://www-cs-faculty.stanford.edu/\~knuth/taocp.html}{The Art
    Of Computer Programming (TAOCP)}.  It is something I've looked up
    to ever since my time as a graduate student at Cornell.


The  Emacspeak Speech Odyssey outlined in this paper is, in some
small measure, my own personal
experience of the sentiments he expresses.

--T. V. Raman,  San Jose, CA, August 1, 2024.
\section{References}
\label{sec:org63ce41b}

\begin{enumerate}
\item \href{https://www.drdobbs.com:443/user-interface-a-means-to-an-end/184410453}{User Interface is a means to an end, DDJ 1997}.
\item \href{https://www.gnu.org/s/emacs/}{GNU Emacs}
\item \href{https://en.wikipedia.org/wiki/TeX}{Knuth's \TeX{}}
\item \href{https://emacspeak.blogspot.com/2022/12/aster-spoken-math-on-emacspeak-audio\_21.html}{Audio System For Technical Readings}
\item \href{https://tvraman.github.io/emacspeak//web/releases/release-3.0.html}{Announcing Emacspeak: April 1995}
\item \href{https://emacspeak.sourceforge.net/turning-twenty.html}{Emacspeak At Twenty}
\item \href{http://www.cs.washington.edu/htbin-post/mvis/mvis?ID=636}{The Web, The Way You Want.  Distinguished Lecture Series, UW Oct 2007}
\item \href{https://emacspeak.sourceforge.net/raman/publications/specialized-browsers/}{Specialized Browsers}
\item \href{https://tvraman.github.io/emacspeak/web/01-gemini.ogg}{An Ode To Emacspeak: The Best Is Yet To Come}
\item \href{https://github.com/dectalk/dectalk.git}{Software Dectalk on Github}
\item \href{https://raw.githubusercontent.com/tvraman/emacspeak/master/servers/software-dtk/Readme.org}{Dectalk  setup instructions}
\item \href{https://emacspeak.blogspot.com/2018/06/effective-suggest-and-complete-in-eyes.html}{Effective Suggest And Complete In An Eyes-free Environment}
\item \href{https://asdf.common-lisp.dev/asdf.html}{Common Lisp: asdf}
\item \href{https://www.quicklisp.org/}{Common Lisp: Quicklisp}
\item \href{https://emacspeak.blogspot.com/2015/12/soundscapes-on-emacspeak-audio-desktop.html}{Soundscapes on the Emacspeak Audio Desktop}
\item \href{https://en.wikipedia.org/wiki/REST}{RESTful Web}
\item \href{https://emacspeak.blogspot.com/2023/09/emacs-ergonomics-dont-punish-your.html}{Ergonomic keybindings}
\item \href{https://github.com/alols/xcape}{Minimize chording with XCape}
\item \href{https://emacspeak.blogspot.com/2022/10/learn-smarter-by-taking-rich-hypertext.html}{Learn Smarter By Taking Rich Hypertext Notes}
\item \href{https://idlewords.com/talks/website\_obesity.htm}{Tag Soup, Scripts And Obfuscation: How The Web Was Broken}
\item \href{https://github.com/eafer/rdrview}{Readable Web Pages: RDRView}
\item \href{https://emacspeak.blogspot.com/2024/03/updated-smart-media-selector-for-audio.html}{smart media selectors}
\item \href{https://research.google/blog/beyond-web-20/?hl=in\&m=1}{Beyond Web 2.0, Communications Of The ACM, 2009}
\item \href{https://tvraman.github.io/emacspeak/manual/URL-Templates.html}{Emacspeak Manual: URL Templates}
\item \href{http://emacspeak.blogspot.com/2007/07/emacspeak-and-beautiful-code.html}{Beautiful Code}   An overview of the Emacspeak architecture, O'Reilly Media, 2007.
\item \href{https://www-cs-faculty.stanford.edu/\~knuth/taocp.html}{The Art Of Computer Programming (TAOCP)}
\end{enumerate}
\end{document}
